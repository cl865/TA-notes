\documentclass[12pt]{article}
\usepackage{amsmath}
\usepackage{graphicx}
\usepackage{hyperref}
\usepackage[latin1]{inputenc}
\usepackage{enumitem}
\usepackage[margin=0.5in]{geometry}
\usepackage[LGRgreek]{mathastext}
\usepackage{framed}
\setlist[itemize]{noitemsep, topsep=0pt}
\begin{document}

\begin{center}
\textbf{ILRST/STSCI 2100 Discussion 4: Ch. 6 Probability}
\end{center}

\noindent \textbf{Probability:} Numerical quantity that expresses the likelihood of an event. \\
Can be thought of the long-term relative frequency of an event happening.\\

\noindent \textbf{Sample space:} All possible outcomes for an experiment.\\
Ex. What is the sample space of flipping two coins?\\

\noindent \textbf{Event:} Collection of outcomes with a designated feature. It is a subset of the sample space.\\
Ex. When flipping two coins, what are the possible events of getting at least one tail?\\

\noindent Can use a tree diagram to keep track of sample space and events.\\

\noindent \textbf{Subset:} Denoted by $\subset$\\
\noindent Ex. If $B = \{1,2,3,4,5,6,7\}$, $A = \{2,4,6\}$ and $C = \{6,7,8\}$, is $A \subset B$? Is $C \subset B$?\\

\noindent \textbf{Complement:} Denoted by $A^{C}$, any event such that $A$ does not occur\\
\noindent Ex. Suppose $A$ is the event of flipping heads, then what is $A^{C}$?\\

\noindent \textbf{Union:}  Given two sets $A$ and $B$, $A \cup B$ are the elements that fall in either $A$, $B$ or both. \\
Ex. A = \{apple, orange, banana\} and B = \{orange, watermelon, pear\}. What is $A \cup B$?\\

\noindent \textbf{Intersection:} Given two sets $A$ and $B$, $A \cap B$ are the elements that fall in both $A$ and $B$. \\
Ex. With A and B given above, what is $A \cap B$?\\

\noindent With numbers, if $A = \{1, 4, 5, 7, 8, 9\}$ and $B = \{1, 2, 6, 7, 10\}$, then, \\
$A \cup B = \{1, 2, 4, 5, 6, 7, 8, 9, 10\}$ \\
$A \cap B = \{1, 7\}$ \\
$A \cap B^{C} = \{4, 5, 8, 9\}$ \\
$A^{C} \cap B = \{2, 6, 10\}$ \\
Notice that $A \cap B + A \cap B^{C} = A$ and similarly, $A \cap B + A^{C} \cap B = B$. \\

\noindent Can also show graphically with Venn Diagrams \\

\noindent \textbf{Probability Axioms}
\begin{enumerate}[noitemsep]
\item $P(A) > 0$\\
\item $P(S) = 1$\\
Ex. Suppose P(A) = 0.3, P(B) = 0.2, and P(C) = 0.4. Are A, B, and C the only elements in the set?
\item $P(A_1\ or\ A_2\ or\ ...\ or\ A_n) = P(A_1)+P(A_2) + ... + P(A_n)$, assuming that no two events can happen at the same time.
\end{enumerate}

\noindent \textbf{Implications}
\begin{enumerate}[noitemsep]
\item Complement Rule: $P(A) = 1 - P(A^C)$ \\
Ex. When rolling a weighted die, the probability of it landing a "2" is 1/3. What is the probability of not landing a "2"?
\item Addition Rule: $P(A\ or\ B) = P(A) +P(B) - P(A \cap B)$ \\
Ex. If $P(A^C) = 0.3$, $P(B) = 0.2$ and $P(A \cap B) = 0.1$, then what is P(A or B)?
\end{enumerate}
\noindent \textbf{Conditional Probability}\\
\noindent "Probability of A given B" \\
$P(A|B) = \cfrac{P(A \cap B)}{P(B)}$\\
Multiplication rule of probability: $P(A \cap B) = P(B)*P(A|B) = P(A)*P(B|A)$\\

\noindent Can be visualized using a Venn Diagram\\
\noindent Applicable for contingency tables


\end{document}