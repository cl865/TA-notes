\documentclass[12pt]{article}
\usepackage{amsmath}
\usepackage{graphicx}
\usepackage{hyperref}
\usepackage[latin1]{inputenc}
\usepackage{enumitem}
\usepackage[margin=0.5in]{geometry}
\usepackage[LGRgreek]{mathastext}
\usepackage{framed}
\setlist[itemize]{noitemsep, topsep=0pt}
\begin{document}
\title{5. Probability}
\begin{center} \textbf{ILRST/STSCI 2100: Discussion 5: Ch. 6 Probability (Review) and Ch. 7 Random Variable }
\end{center}

\noindent \textbf{Review from last week:} \\

Addition rule: P(A or B) $= P(A) + P(B) - $P(A and B)\\

Multiplicative rule: P(A and B) $= P(A|B) \times P(B) = P(B|A)\times P(A)$\\

Complement: $P(A^c) = 1 - P(A)$ \\

Conditional probability: $P(A|B) = \frac{P(A \text{ and } B)}{P(B)}$\\

\noindent Some helpful "formulas" - can derive with drawing Venn Diagrams: \\

\noindent P(A) = P(A and $B^c)$ + P(A and B), similarly, P(B) = P($A^c$ and B) + P(A and B)\\

\noindent P(A or B or C) = P(A) + P(B) + P(C) - P(A and B) - P(A and C) - P(B and C) + P(A and B and C)\\

\noindent \textbf{Independence} : $P(A|B) = P(A)$ and $P(B|A) = P(B)$ \\

\noindent Note: The multiplication rule: $P(A \cap B) = P(A|B) * P(B)$ \\ 

\noindent So $P(A \cap B) = P(A) * P(B)$ when $A$ and $B$ are independent. \\

\noindent Errors typically arise when you assume $A$ and $B$ are independent and do calculations as if they are independent events. Always assume dependence, unless you have shown otherwise. \\

\noindent Note: Independent events vs mutually exclusive events:
\begin{itemize}
	\item Mutually exclusive events are not independent
	\begin{itemize}
		\item Flipping a head and a tail are mutually exclusive (they can't both happen at the same time). So, you if flip a head, then you know that you didn't flip a tail. That is, prior knowledge (head), helped you make a statement about probability of flipping tail, meaning these are not independent events. 
	\end{itemize}
	\item Independent events cannot be mutually exclusive
		\begin{itemize}
		\item If two events are independent, $P(A \cap B) = P(A)*P(B)$ is nonzero, implying both can happen at the same time, meaning not mutually exclusive. 
	\end{itemize}
\item Exception, independent events appear mutually exclusive when $P(A)$ or $P(B)$ is $0$. \\
\end{itemize}

\noindent \textbf{Random Variables}\\
Either discrete (countable) or continuous (any number on the number line)\\
Denoted with a capital letter, ex. "X"\\

\noindent Random variables are associated with a probability function\\
\underline{Discrete:} probability mass function (pmf)\\

f(x) = P(X = x) denotes the probability of observing x \\

$ 0 \leq f(x) \leq 1$ and $\sum_x f(x) = 1$

Ex. Let X be a random variable that is the number of tails when you flip 3 coins. What is the p.m.f of X?\\

\noindent \underline{Continuous:} probability density function (p.d.f) \\

$f(x) \geq 0$ for all x\\

$\int_{-\infty}^{+\infty} f(x) dx = 1$ : area under the entire curve is 1.\\

For a continuous (only!) variable, $P(a \leq X \leq b) = P(a < X < b)$, can be thought of as the area under the curve bounded by a range. And this is because P(X = x) = 0 for a continuous variable.\\

Ex. What is the probability that a car on the road is going 65 mph?\\

\noindent \textbf{Expected value} \\

The center/mean of a random variable $X$, $E(X) = \mu$, in the discrete case, this is: $E(X) = \sum_{x} xf(x) = \mu$ \\

Variance of a discrete random variable: $Var(X) = \sigma^2 = E[(X -\mu)^2] = \sum_x (x - \mu)^2f(x)$ Given the following p.m.f, find the variance:\\

Standard Deviation = Square root of variance\\

\noindent \textbf{Practice Problems:}
Given P(A) = 0.5; P(B) = 0.2; P(C) = 0.2
\begin{enumerate}
\item Are A, B, and C the only elements in the sample space?
\item If A and B are independent, what is $P(A \cap B)?$
\item If A and B are mutually exclusive, what is $P(A \cup B)?$
\item Now suppose $P(A \cap B) = 0.2$. What is $P(A|B)?$

\end{enumerate}

\noindent \textbf{Answers:}
\begin{enumerate}
\item No, we can find P(A)+P(B)+P(C) = 0.9, which isn't equal to one, so there must be more element(s) in the sample space. 
\item Since $A$ and $B$ are independent, $P(A \cap B) = P(A)*P(B) = 0.5*0.2 = 0.1$
\item Using the general additional rule: $P(A \cup B) = P(A) + P(B) - P(A\cap B)$ 
Since $A$ and $B$ are mutually exclusive, $P(A \cup B) = 0$, so $P(A \cup B) = P(A) + P(B) - 0 = 0.2 + 0.5 -0 = 0.7$. 
\item $P(A|B) = \cfrac{P(A \cap B)}{P(B)} = \cfrac{0.2}{0.2} = 1$
\end{enumerate}
\end{document}