\documentclass[12pt]{article}
\usepackage{amsmath}
\usepackage{graphicx}
\usepackage{hyperref}
\usepackage[latin1]{inputenc}
\usepackage{enumitem}
\usepackage[margin=0.5in]{geometry}
\usepackage[LGRgreek]{mathastext}
\usepackage{framed}
\setlist[itemize]{noitemsep, topsep=0pt}
\title{11. Categorical Data}

\begin{document}
\begin{center}
\textbf{Ch. 12 Analysis of Categorical Data and Goodness of Fit Tests}
\end{center}

\noindent \textbf{Goodness of Fit}\\
\noindent Given a sample with $k$ categories, tests how well the sample fits the hypothesized distribution. \\
Assumptions:\\
--Random sample and all expected counts are greater than 5.
\begin{enumerate}
\item
$H_{0}: p_{i}=p_{i0}$ for all i=1,2,...,k
\item
$H_{a}: p_{i} \neq p_{i0}$ for at least one i=1,2,...,k \\
Note: the alternative is always two-sided
\item
Test statistic: \\
$\chi_{obs}^{2}=\sum_{i=1}^{k}\cfrac{(O_{i}-E_{i})^{2}}{E_{i}}$
\item
Find p-value\\
$pvalue = P(\chi^{2}>\chi_{obs}^{2})$, with $df = k-1$\\
pvalue is always the area in the upper tail region
\end{enumerate}

\noindent \textbf{Ex.} Rolling a die 36 times, test the null hypothesis that the a die is a fair die given the observation:\\
1: 4 times; 2: 7 times; 3: 5 times; 4: 6 times; 5: 8 times; 6: 6 times \\
\\
\noindent There are 6 categories,\\
$H_{0}: p_{i} = 1/6$ for $i=1, 2, ..., 6$\\ 
$H_{a}: p_{i} \neq 1/6$ for at least 1 $i=1, 2, ..., 6$\\
\\
\noindent If this is a fair die, we would expect 6 throws for each face.\\
\\
$\chi_{obs}^{2} = \cfrac{(4-6)^{2}}{6}+\cfrac{(7-6)^{2}}{6}+\cfrac{(5-6)^{2}}{6}+\cfrac{(6-6)^{2}}{6}+\cfrac{(8-6)^{2}}{6}+\cfrac{(6-6)^{2}}{6}=1.67$\\
\\
The $df=6-1=5$, $pvalue = P(\chi^{2}>1.67)$, using Table 8, $pvalue > .1$\\
Since the pvalue is less than $\alpha$, we fail to reject the null hypothesis and conclude that the die is fair. \\

\noindent \textbf{Chi-square test of homogeneity in a contingency table:}\\

\begin{center}
\begin{tabular}{ c c c c c c c c}
 Year & CALS & Engineering & ILR & AS&Total\\ 
 Freshman & 30 & 40 &20 &40 &130 \\  
 Sophomore & 20 & 50 &30 &40&140 \\
 Junior & 30 & 20 &35 &55 &140 \\  
 Senior & 35 & 25 &15 &30 & 105 \\
 \hline
 Total &115 & 135 &100 & 165 &515
\end{tabular}
\end{center}

\noindent Test at 5\% significance level that the distribution of student's affiliated college is evenly distributed among class year. \\

\noindent $H_{0}: p_{1j}=p_{2j}=...=p_{rj}$, for all j\\

\noindent $H_{a}:$ at least one $p_{ij}$ differs

\noindent First generate the expected values table using the formula: $E_{ij} = \cfrac{RowTotal * ColumnTotal}{GrandTotal}$\\


\begin{center}
\begin{tabular}{ c c c c c c c c}
 Year & CALS & Engineering & ILR & AS&Total\\ 
 Freshman & 29 & 34 &25 &42 &130 \\  
 Sophomore & 31 & 37 &27 &45&140 \\
 Junior & 31 & 37 &27 &45 &140 \\  
 Senior & 23 & 28 &20 &34 & 105 \\
 \hline
 Total &114 & 136 &99 & 166 &515
\end{tabular}
\end{center}

\noindent Due to rounding, the row and column totals may be slightly off. Note that all expected values are greater than 5. \\

\noindent Next, calculate $\chi^{2}_{obs}$.\\
\\
$\chi^{2}_{obs} = \cfrac{(30-29)^{2}}{29}+\cfrac{(40-34)^{2}}{34}+\cfrac{(20-25)^{2}}{25}+...$, repeat for each cell of the table, so in total, you should have the summation of 16 fractions. \\

\noindent Next, find p-value. The $df = (r-1)(k-1) = (4-1)(4-1) = 9$, where r is the number of rows and k is the number of columns. \\

\noindent Lastly, compare p-value to alpha. \\

\noindent \textbf{The chi-square test for independence:}\\
\begin{itemize}
\item
$H_{0}:$ The two categorical variables are independent.
\item
$H_{a}:$ The two categorical variables are not independent.
\item
\noindent Test statistic: $\chi^{2}_{obs}$ (same formula)
\item
\noindent pvalue = $P(\chi^{2}_{obs} > \chi^{2}_{obs})$, and df = (number of rows - 1)*(number of columns - 1)
\item
\noindent Compare to $\alpha$.
\end{itemize}



\end{document}


