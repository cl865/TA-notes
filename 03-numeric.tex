\documentclass[12pt]{article}
\usepackage{amsmath}
\usepackage{graphicx}
\usepackage{hyperref}
\usepackage[latin1]{inputenc}
\usepackage{enumitem}
\usepackage[margin=0.5in]{geometry}
\usepackage[LGRgreek]{mathastext}
\usepackage{framed}
\setlength{\parskip}{1em}
\setlist[itemize]{noitemsep, topsep=0pt}
\begin{document}

\begin{center}
\textbf{ILRST/STSCI 2100 Discussion 3: Ch. 4 Numerical Methods}
\end{center}

\noindent \textbf{Review summations}
\begin{itemize}
\item Suppose $c$ is a constant, then $\sum_{i=1}^{n} c = n*c$\\

\indent Ex. $\sum_{i=1}^3 2 = 2+2+2 = 3*2 = 6$ \\

\item $\sum_{i=1}^{n}( a_{i}+b_{i}) = \sum_{i=1}^{n}a_{i} + \sum_{i=1}^{n} b_{i}$ \\

\indent Ex. Suppose $a = \{1,2,3,4\}$ and $b=\{5,6,7,8\}$ \\
\indent $\sum_{i=1}^{4}(a_{i}+b_{i}) = \sum_{i=1}^{4}a_{i}+\sum_{i=1}^{4}b_{i}= 10 + 26 = 36$ \\
\item Similarly, $\sum_{i=1}^{n} a_{i} - b_{i} = \sum_{i=1}^{n} a_{i} - \sum_{i=1}^{n} b_{i}$ \\
\item \textbf{Note:} $\sum a_{i}b_{i} \neq \sum a_{i} \sum b_{i}$ 

\end{itemize}

\noindent \textbf{Mean}

$\bar{x} = \cfrac{\sum_{i=1}^{n} x_{i}}{n}$ \hspace{8pt} "sum all the values and divide by the number of values you have"\\

\noindent Ex. $x=\{3,1,8,4,6,3,8,3\}$ Find the mean of x.\\

$\begin{aligned}
\bar{x} = \cfrac{\sum x_{i}}{n} &= \cfrac{3+1+8+4+6+3+8+3}{8} \\
&=\cfrac{36}{8} = 4.5
\end{aligned}$

\noindent \textbf{Trimmed Mean} \\
Finding the mean after removing some percent of large values and small values. Make sure data is first sorted. The purpose of trimming is to remove possible outliers\\
\noindent Ex. $x=\{1,2,3,4,5,6,7,8,9\}$ Find the 10\% trimmed mean of $x$. 
\begin{enumerate}
\item The sample size is 9, as there are 9 values in $x$. 
\item 10\% of $9$ is 0.9, we round to the nearest integer, so we remove 1 item from the high end and 1 item from the small end.
\item After trimming, the set is \{2,3,4,5,6,7,8\}
\item Calculate the mean: $\cfrac{2+3+4+5+6+7+8}{7} = 35/7$
\end{enumerate}

\noindent \textbf{Median} \\
The middle value. The median is a robust statistic, meaning it is resistant to outliers and changes in small portions data. \\
\noindent This measure is preferred over the mean when there are outliers or skewed data. \\
Ex. For $x=\{1,2,3,4,5\}$, the median is 3. \\
\indent \hspace{5pt}For $x=\{1,2,3,4,5,6\}$, the median is the average of the two "middle terms", 3.5 \\

\noindent \textbf{Quartiles} \\
Divide the data into fourths, Q1 = 25th percentile, Q2 = median = 50th percentile, Q3 = 75th percentile \\
$IQR = Q3 - Q1$ \\
Mild outlier: falls outside the range of $Q1-1.5*IQR$ and $Q3+1.5*IQR$ \\
Extreme outlier: falls outside the range of $Q1-3*IQR$ and $Q3+3*IQR$ \\

\noindent \textbf{Range:}\\
Measure of variation,  
$max - min$

\noindent \textbf{Standard Deviation:} \\
\vspace{5mm}
$s=\sqrt{\cfrac{\sum(x_{i}-\bar{x})^{2}}{n-1}}$\\
The sample variance is denoted as $s^{2}$ \\
Note that $s_{xx} = \sum(x_{i}-\bar{x})^{2} = \sum_{i = 1}^n x_{i}^2-\frac{(\sum_{i=1}^nx_i)^2}{n}$  \\
Also note: $\sum_i x_i^2 \neq (\sum_i x_i)^2$

\noindent\textbf{z-score:} \\
\noindent z score = $\cfrac{value-mean}{standard deviation}$ \\

\noindent Calculating z-scores is also called standardization

\noindent \textbf{Parameter vs Statistic:} \\
A parameter is a value about the population, and a statistic is a value about the sample.
\begin{itemize}
\item Mean: $\mu$ vs $\bar{x}$
\item Variance: $\sigma^{2}$ vs $s^{2}$
\item Proportion: $p$ vs $\hat{p}$
\end{itemize}



\end{document}